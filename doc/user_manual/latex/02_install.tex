\chapter{Installation Instructions}

\section{Prerequisites}

PerfExpert is based on other tools so that installation of PerfExpert requires that they be installed. These tools are:

\begin{itemize}
	\item \textbf{PAPI} (\url{http://icl.cs.utk.edu/papi/software/}): PAPI is required to measure hardware performance metrics like cache misses, branch instructions, etc. The PAPI installation is mostly straightforward: download, \texttt{./configure}, \texttt{make}, and \texttt{make install}. If your Linux kernel version is 2.6.32 or higher, then PAPI will mostly likely use \texttt{perf\_events}. Recent versions of PAPI (v3.7.2 and beyond) support using \texttt{perf\_events} in the Linux kernel. However, if your kernel version is lower than 2.6.32, then you would require patching the kernel with either \texttt{perfctr}\footnote{\url{http://user.it.uu.se/~mikpe/linux/perfctr/2.6/}} or \texttt{perfmon}\footnote{\url{http://perfmon2.sourceforge.net/}}.
	\item \textbf{HPCToolkit} (\url{http://hpctoolkit.org/software.html}): HPCToolkit is a tool that works on top of PAPI. HPCToolkit is used by PerfExpert to run the program multiple times with specific performance counters enabled. It is also useful for correlating addresses in the compiled binary back to the source code.
	\item \textbf{Java Virtual Machine} and the \textbf{Java Development Kit}\\(\url{http://www.oracle.com/technetwork/java/javase/downloads/}).
	\item \textbf{ROSE Compiler} (\url{http://rosecompiler.org/}): ROSE is the framework PerfExpert uses to manipulate the application�s source code. It is required only if you wish to compile PerfExpert with performance optimization capability.
	\item \textbf{Apache Ant} (\url{http://ant.apache.org/bindownload.cgi}): it is required only to compile PerfExpert
	\item \textbf{SQLite} version 3 (\url{http://www.sqlite.org/}): PerfExpert uses a SQLite database to store suggestions for bottleneck remediation and other information for automatic optimization.
	\item \textbf{GNU Multiple Precision Arithmetic Library} (\url{http://gmplib.org/}): MACPO, one of the tools which compose PerfExpert requires this package.
	\item \textbf{Google SparseHash} (\url{https://code.google.com/p/sparsehash/}):\\MACPO, one of the tools which compose PerfExpert requires this package.
\end{itemize}

\subsection{Installing Prerequisites}

We provide a \texttt{INSTALL} script in the \texttt{perfexpert-externals-X.X.tar.gz} package. This script is an example of how to install some of PerfExpert prerequisites. To install PAPI, Java Virtual Machine, Java Development Kit, and SQLite follow the instructions of your Linux distribution.

\nicebox{CAUTION:}{The \texttt{INSTALL} script should be modified to fit for your environment. Do not run it out-of-the-box!}

\subsubsection{Apache Ant}

The PerfExpert externals package contains another package named \texttt{apache-ant-1.9.1-bin.tar.gz}. This is the binary version of Apache Ant. As PerfExpert only needs Apache Ant during compilation there is no need to install this package in your system. Thus, you should only decompress it. To do that, run the following commands:

\begin{verbatim}
$ tar -xzvf ./apache-ant-1.9.1-bin.tar.gz
$ export ANT_HOME=`pwd`/apache-ant-1.9.1
\end{verbatim}

\noindent these commands will decompress Apache Ant and set the \texttt{ANT\_HOME} environment variable to reflect the \texttt{PATH} where such package has been decompressed.

\subsubsection{HPCToolkit}

All HPCToolkit prerequisites are provided in the package \texttt{hpctoolkit-externals-5.3.2-r4197.tar.gz}. The PerfExpert externals package provides both HPCToolkit and HPToolkit prerequisites packages. To decompress and install the HPCToolkit prerequisites package run the following commands:

\begin{verbatim}
$ tar -xzvf ./hpctoolkit-externals-5.3.2-r4197.tar.gz
$ cd hpctoolkit-externals-5.3.2-r4197/
$ ./configure
$ make
\end{verbatim}

\nicebox{NOTICE:}{The HPCToolkit externals package does not has to be installed in your system, HPCToolkit will take care of this. However, it should be compiled and available during HPCToolkit compilation and installation process.}

After compiling the prerequisites for HPCToolkit, the HPCToolkit package can be installed. For that, run the following commands:

\begin{verbatim}
$ tar -xzvf ./hpctoolkit-5.3.2-r4197.tar.gz
$ cd hpctoolkit-5.3.2-r4197/
$ ./configure --prefix=WHERE_YOU_WANT_TO_INSTALL_IT \
    --with-externals=PATH_TO_HPCTOOLKIT_EXTERNALS_PACKAGE \
    --with-papi=PATH_TO_PAPI_INSTALLATION
$ make install
\end{verbatim}

\subsubsection{ROSE}

ROSE has several prerequisites. One of the most important one is the compiler. You should have \texttt{GCC} version 4.4 to compile ROSE. It also requires a specific version of the Boost library (1.47). We do provide the Boost library package, but to install \texttt{GCC} version 4.4 you should follow the instruction of your Linux distribution. To compile and install the Boost library version 1.47 run the following commands:

\nicebox{NOTICE:}{We had successfully compiled and installed ROSE using \texttt{GCC} 4.6, however according to the ROSE documentation this compiler is not supported. If you want to use this version of \texttt{GCC} do it carefully.}

\begin{verbatim}
$ tar -xzvf ./boost_1_47_0.tar.gz
$ cd boost_1_47_0/
$ ./bootstrap.sh
$ ./b2 install --prefix=${INSTALL_DIR}
\end{verbatim}

After installing successfully the Boost library run the following commands to compile and install ROSE:

\begin{verbatim}
$ tar -xzvf ./rose-0.9.5a-without-EDG-20584.tar.gz
$ cd rose-0.9.5a-20584/
$ mkdir _build
$ cd _build
$ ../configure --prefix=WHERE_YOU_WANT_TO_INSTALL_IT --with-boost=WHERE_BOOST_IS \
    --disable-tutorial-directory --disable-projects-directory --disable-opencl \
    --disable-tests-directory --disable-php --disable-cuda --disable-binary-analysis \
    --disable-java --with-gomp_omp_runtime_library=WHERE_LIBGOMP_IS \
    --libexecdir=/tmp --bindir=/tmp --sbindir=/tmp
$ make install
\end{verbatim}

\nicebox{NOTICE:}{The \texttt{configure} command we shown above avoids the compilation and installation of several features of ROSE. We do that to minimize the time required to compile ROSE. Moreover, some binaries that ROSE installs and we do not need them have been set to be installed into the \texttt{/tmp} directory. It is safe to clean the temporary directory after installing ROSE.}

\nicebox{ATTENTION:}{If you want to PerfExpert be able to optimize OpenMP applications you should set the \texttt{--with-gomp\_omp\_runtime\_library} argument to where \texttt{GCC} has installed \texttt{libgomp.so}.}

\subsubsection{GNU Multiple Precision Arithmetic Library}

Many Linux distributions already have the GNU Multiple Precision Arithmetic Library installed. To check if your system has it try to locate the \texttt{libgmp.so} file. If your system does not has the GNU Multiple Precision Arithmetic Library, run the following commands to install it:

\begin{verbatim}
$ tar -xzvf ./gmp-5.1.2.tar.gz
$ cd gmp-5.1.2/
$ ./configure --prefix=WHERE_YOU_WANT_TO_INSTALL_IT
$ make
$ make check
$ make install
\end{verbatim}

\nicebox{NOTICE:}{The GNU Multiple Precision Arithmetic Library is required to compile and execute MACPO, one of the tools which is part of PerfExpert. If you do not want to install MACPO you do not need to install the GNU Multiple Precision Arithmetic Library.}

\subsubsection{Google SparseHash}

To install Google SparseHash you should run the following commands:

\begin{verbatim}
$ tar -xzvf ./sparsehash-2.0.2.tar.gz
$ cd sparsehash-2.0.2/
$ ./configure --prefix=WHERE_YOU_WANT_TO_INSTALL_IT
$ make install
\end{verbatim}

\nicebox{NOTICE:}{Google SparseHash is required to compile and execute MACPO, one of the tools which is part of PerfExpert. If you do not want to install MACPO you do not need to install the Google SparseHash.}

\section{Downloading PerfExpert}

PerfExpert is an open-source project. Funding to keep researchers working on PerfExpert depends on the value of this tool to the scientific community. For that reason, it is really important to know where and who are using our tool. We would really appreciate it if you could send us a message (\href{mailto:agomez@tacc.utexas.edu}{\texttt{agomez@tacc.utexas.edu}}) telling us the institution (name and country) you are planning to install and test PerfExpert at.

The PerfExpert source code can be downloaded from the registration page:\\ \url{http://www.tacc.utexas.edu/perfexpert/registration}.

We encourage people to have a look on the wiki page, send patches, and raise issues on PerfExpert's page on GitHub: \url{https://github.com/TACC/perfexpert}.

\section{Setting Up PerfExpert}

If you have downloaded PerfExpert from our version control system, you may have already noticed that there is no configure script available on the source code tree. To generate it run the following command:

\begin{verbatim}
$ ./autogen.sh
\end{verbatim}

\texttt{autogen.sh} requires the following packages available:

\begin{itemize}
	\item M4 version 1.4.13 or newer (\url{ftp://ftp.gnu.org/gnu/m4/})
	\item Autoconf version 2.63 or newer (\url{ftp://ftp.gnu.org/gnu/autoconf/})
	\item Automake version 1.11.1 or newer (\url{ftp://ftp.gnu.org/gnu/automake/})
	\item Libtool version 2.2.6b or newer (\url{ftp://ftp.gnu.org/gnu/libtool/})
\end{itemize}

PerfExpert comes with a \texttt{Makefile}-base source code tree to automate the entire installation process. Thus the compilation and installation of PerfExpert is similar to any other GNU package:

\begin{verbatim}
$ ./configure
$ make
$ make install
\end{verbatim}

Optionally, you may want to run the set of test we have included into PerfExpert. To do so, just run ``\texttt{make check}'' after compiling your code.

If any of the prerequisites of PerfExpert are not on the \texttt{PATH} or \texttt{LD\_LIBRARY\_PATH}, you can specify the right locations of such files. For that, have a look on the configure script help using the following command:

\begin{verbatim}
$ ./configure --help
\end{verbatim}

A typical command line to run \texttt{configure} may looks like this:

\begin{verbatim}
$ ./configure --prefix=WHERE_YOU_WANT_TO_INSTALL_IT --with-rose=WHERE_ROSE_IS \
    --with-jvm=WHERE_JVM_IS --with-papi=WHERE_PAPI_IS --with-apache-ant=WHERE_ANT_IS
\end{verbatim}

Optionally, you may want to run the set of test we have included into PerfExpert. To do so, just run ``\texttt{make check}'' after compiling your code.

In case you have any problem installing PerfExpert, send us an email\\ (\href{mailto:agomez@tacc.utexas.edu}{\texttt{agomez@tacc.utexas.edu}}) or use our mailing list: \href{mailto:perfexpert-subscribe@lists.tacc.utexas.edu }{\texttt{perfexpert-subscribe@lists.tacc.utexas.edu}}.

\section{Characterizing your Machine}

During the installation process, PerfExpert will run a set of benchmark applications to characterize your machine. This characterization is used to analyze the performance of the applications you want to optimize. For that reason, you should be sure the PerfExpert installation is run on a machine of the same kind you are planning to run the production code on. If it is not possible, you should run the benchmark application and generate the characterization file manually and move it to the directory where PerfExpert has been installed. To do that you should execute the \texttt{hound} command (which is available inside the \texttt{bin} directory of PerfExpert installation) and save the output of this command to a file named \texttt{machine.properties} inside the \texttt{etc} directory where PerfExpert is installed.

\nicebox{ATTENTION:}{This is one of the most important and sensible steps of PerfExpert installation process. Be sure the content of this file reflects the characteristics of the machine where you want analyze the performance of applications. If the content of the \texttt{machine.properties} file is not accurate all the analysis, recommendations for optimization and also automatic optimization PerfExpert will do on your application could not improve it's performance.}

\section{Testing PerfExpert Installation}

We provide a set of tests you may want to run before using PerfExpert to optimize your applications. To run these tests you should run the following command inside the PerfExpert build tree:

\begin{verbatim}
$ make check
\end{verbatim}

It is normal that one of the tests (\texttt{mpi\_stampede}) fails since it has been designed to be executed only on Stampede.
